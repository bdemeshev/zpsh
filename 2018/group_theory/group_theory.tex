\documentclass[a4paper, 12pt]{article}


\usepackage{mathrsfs}

\usepackage{amscd}
\usepackage[paper=a4paper,
top=2cm, bottom=2cm, left=2cm, right=2cm, includefoot]{geometry} % размер листа бумаги


\usepackage{tikz} % картинки в tikz

\usetikzlibrary{calc}
\usepackage{microtype} % свешивание пунктуации

\usepackage{floatrow} % для выравнивания рисунка и подписи
\usepackage{caption} % для пустых подписей

\usepackage{array} % для столбцов фиксированной ширины

\usepackage{indentfirst} % отступ в первом параграфе

\usepackage{sectsty} % для центрирования названий частей
\allsectionsfont{\centering}

\usepackage{amsmath, amsfonts} % куча стандартных математических плюшек

\usepackage{comment} % для комментариев

\usepackage{multicol} % текст в несколько колонок

\usepackage{lastpage} % чтобы узнать номер последней страницы

\usepackage{enumitem} % дополнительные плюшки для списков
%  например \begin{enumerate}[resume] позволяет продолжить нумерацию в новом списке

\usepackage{booktabs}

\usepackage{url} % для вставки интернет-ссылок

\usepackage{fontspec}
\usepackage{polyglossia}

\setmainlanguage{russian}
\setotherlanguages{english}

% download "Linux Libertine" fonts:
% http://www.linuxlibertine.org/index.php?id=91&L=1
\setmainfont{Linux Libertine O} % or Helvetica, Arial, Cambria
% why do we need \newfontfamily:
% http://tex.stackexchange.com/questions/91507/
\newfontfamily{\cyrillicfonttt}{Linux Libertine O}

\AddEnumerateCounter{\asbuk}{\russian@alph}{щ} % для списков с русскими буквами
\setlist[enumerate, 2]{label=\asbuk*),ref=\asbuk*}

\DeclareMathOperator{\Var}{Var}
\DeclareMathOperator{\E}{\mathbb{E}}

\let\P\relax
\DeclareMathOperator{\P}{\mathbb{P}}
\def\cN{\mathcal{N}}

\usepackage{fancyhdr} % весёлые колонтитулы
\pagestyle{fancy}
\lhead{Теория групп}
\chead{}
\rhead{ЗПШ-2018}
\lfoot{}
\cfoot{}
\rfoot{\thepage/\pageref{LastPage}}
\renewcommand{\headrulewidth}{0.4pt}
\renewcommand{\footrulewidth}{0.4pt}


\begin{document}

\section{Анонс}

За кубиком Рубика и игрой Пятнашки маячит хвостик большого раздела математики под названием теория групп. Серьёзные дяденьки и тётеньки применяют теорию групп в физике и химии. 

А мы будем собирать кубик Рубика, рисовать простые группы, поймём, что $a$ умножить на $b$ не всегда равно $b$ умножить на $a$, найдём разрешимые и неразрешимые позиции в головоломках, увидим что-то общее между умножением, надеванием носков и переворачиваниями матраса :)

\section{Презентация}

Презентация длится 10 минут, три дубля презентации для разных школьников. 
Объявленная аудитория 9-11 класс.

Знак «3» будет означать действие «умножь задуманное число на три». Тогда по смыслу тождество
\[
3 \cdot 5 = 5 \cdot 3
\]
означает два действия, выполненные в разных порядках.

Пример группы. У бедного студента на левой ноге надет носок. Студент умеет выполнять команды $p$ — переодень носок на другую ногу и $v$ — сними носок, выверни наизнанку и надень на исходную ногу. Рисуем для данного примера диаграмму Кэли, не произнося таких страшных снов.

Вводим нейтральный элемент группы $n$ — ничего не делать, обнаруживаем тождества $ab=ba$, $a^2=n$, $a^3b^2=a$. Мы используем запись $ab$ — сначала действие $a$, потом действие $b$.

Ещё пример группы. Аня, Белла и Вика сидят на стульях 1, 2 и 3. Они умеют выполнять инструкцию $a=(123)$ и $b=(12)$. Инструкция $(345)$ означает, что тот, кто сидел на месте 3 садится на место 4; тот, кто сидел на месте 4, — на место 5; и тот, кто сидел на месте 5, — на место 3.

Правда ли, что $ab=ba$?

Кубик Рубика — тоже группа. Правда ли, что $\text{П}\text{В}=\text{В}\text{П}$? Здесь буква означает вращение соответствующей грани по часовой на $90^{\circ}$.


\newpage
\section{Встреча 1}


Андреев Павел 8 \\
Завалова Ульяна 9 \\
Оводов Александр 11 \\
Алексей Корчагин 6 \\
Ведерникова Юлия 9 \\
Диана ... в \\

\begin{enumerate}

  \item Два выключателя рядом на стене. Два образующих действия: $a$ — переключить оба, $l$ — переключить левый.
    \begin{enumerate}
      \item Нарисуйте схему всех возможных состояний и соедините их стрелочками $a$ и $l$;
      \item В чём смысл составного действия $al$?
      \item Что по-сути означает действие $a^2$?
      \item Сколько всего разных действий, считая исходные образующие и все действия, что можно из образующих получить?
      \item Упростите формулу $a^5l^2a^3$;
      \item Составьте таблицу умножения всех действий. По строке — первое действие, по столбцу — второе.
    \end{enumerate}

    
  \item Солдат умеет выполнять всего один приказ $a$. Единственное образущее действие: $a$ — повернуться вправо на $90^{\circ}$.    
    \begin{enumerate}
      \item Нарисуйте схему всех возможных состояний и соедините их стрелочками $a$;
      \item В чём смысл составного действия $a^{2018}$?
      \item Сколько всего разных действий, считая исходное образующее и все действия, что можно из образующего получить?
      \item Составьте таблицу умножения всех действий. По строке — первое действие, по столбцу — второе.
    \end{enumerate}

  \item У бедного студента на левой ноге надет носок. Студент умеет выполнять команды $p$ — переодень носок на другую ногу и $v$ — сними носок, выверни наизнанку и надень на исходную ногу.     
    \begin{enumerate}
      \item Нарисуйте схему всех возможных состояний и соедините их стрелочками $p$ и $v$;
      \item В чём смысл составного действия $p^{12}v^{2019}$?
      \item Что по-сути означает действие $p^2$?
      \item Сколько всего разных действий, считая исходные образующие и все действия, что можно из образующих получить?
      \item Составьте таблицу умножения всех действий. По строке — первое действие, по столбцу — второе.
    \end{enumerate}

  \item Мы составили три таблицы умножения для действий в группах. Что особенного в каждой строке и в каждом столбце?


\end{enumerate}

Мысль: Обратимость действия в группе приводит к тому, что ни в одной строке, ни в одном столбце нет повторяющихся действий.

\begin{enumerate}[resume]
  \item Если возможно, постройте группу из трёх действий, с таблицей умножения
\begin{tabular}{@{}llll@{}}
\toprule
 & $a$ & $b$ & $c$   \\ \midrule
$a$ & $a$  & $b$ & $c$   \\
$b$ &  & $a$ &    \\
$c$ &  &  &    \\ \bottomrule
\end{tabular}
\end{enumerate}

\end{document}
