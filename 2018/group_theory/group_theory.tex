\documentclass[a4paper, 12pt]{article}


\usepackage{mathrsfs}

\usepackage{amscd}
\usepackage[paper=a4paper,
top=2cm, bottom=2cm, left=2cm, right=2cm, includefoot]{geometry} % размер листа бумаги


\usepackage{tikz} % картинки в tikz
\usepackage{microtype} % свешивание пунктуации

\usepackage{floatrow} % для выравнивания рисунка и подписи
\usepackage{caption} % для пустых подписей

\usepackage{array} % для столбцов фиксированной ширины

\usepackage{indentfirst} % отступ в первом параграфе

\usepackage{sectsty} % для центрирования названий частей
\allsectionsfont{\centering}

\usepackage{amsmath, amsfonts} % куча стандартных математических плюшек

\usepackage{comment} % для комментариев

\usepackage{multicol} % текст в несколько колонок

\usepackage{lastpage} % чтобы узнать номер последней страницы

\usepackage{enumitem} % дополнительные плюшки для списков
%  например \begin{enumerate}[resume] позволяет продолжить нумерацию в новом списке

\usepackage{booktabs}

\usepackage{url} % для вставки интернет-ссылок

\usepackage{fontspec}
\usepackage{polyglossia}

\setmainlanguage{russian}
\setotherlanguages{english}

% download "Linux Libertine" fonts:
% http://www.linuxlibertine.org/index.php?id=91&L=1
\setmainfont{Linux Libertine O} % or Helvetica, Arial, Cambria
% why do we need \newfontfamily:
% http://tex.stackexchange.com/questions/91507/
\newfontfamily{\cyrillicfonttt}{Linux Libertine O}

\AddEnumerateCounter{\asbuk}{\russian@alph}{щ} % для списков с русскими буквами
\setlist[enumerate, 2]{label=\asbuk*),ref=\asbuk*}

\DeclareMathOperator{\Var}{Var}
\DeclareMathOperator{\E}{\mathbb{E}}

\let\P\relax
\DeclareMathOperator{\P}{\mathbb{P}}
\def\cN{\mathcal{N}}

\usepackage{fancyhdr} % весёлые колонтитулы
\pagestyle{fancy}
\lhead{Теория групп}
\chead{}
\rhead{ЗПШ-2018}
\lfoot{}
\cfoot{}
\rfoot{\thepage/\pageref{LastPage}}
\renewcommand{\headrulewidth}{0.4pt}
\renewcommand{\footrulewidth}{0.4pt}


\begin{document}

\section{Презентация}

Презентация длится 10 минут, три дубля презентации для разных школьников. 
Объявленная аудитория 9-11 класс.

Знак «3» будет означать действие «умножь задуманное число на три». Тогда по смыслу тождество
\[
3 \cdot 5 = 5 \cdot 3
\]
означает два действия, выполненные в разных порядках.

Пример группы. У робота андроида на левой ноге надет носок. Робот умеет выполнять команды $a$ — переодень носок на другую ногу и $b$ — сними носок, выверни наизнанку и надень на исходную ногу. Рисуем для данного примера диаграмму Кэли, не произнося таких страшных снов.

Вводим нейтральный элемент группы $n$ — ничего не делать, обнаруживаем тождества $ab=ba$, $a^2=n$, $a^3b^2=a$. Мы используем запись $ab$ — сначала действие $a$, потом действие $b$.

Ещё пример группы. Аня, Белла и Вика сидят на стульях 1, 2 и 3. Они умеют выполнять инструкцию $a=(123)$ и $b=(12)$. Инструкция $(345)$ означает, что тот, кто сидел на месте 3 садится на место 4; тот, кто сидел на месте 4, — на место 5; и тот, кто сидел на месте 5, — на место 3.

Правда ли, что $ab=ba$?

Кубик Рубика — тоже группа. Правда ли, что $\text{П}\text{В}=\text{В}\text{П}$? Здесь буква означает вращение соответствующей грани по часовой на $90^{\circ}$.

\section{Встреча 1}



\end{document}
