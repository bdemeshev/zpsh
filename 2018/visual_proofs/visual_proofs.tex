\documentclass[a4paper, 12pt]{article}


\usepackage{mathrsfs}

\usepackage{amscd}
\usepackage[paper=a4paper,
top=2cm, bottom=2cm, left=2cm, right=2cm, includefoot]{geometry} % размер листа бумаги


\usepackage{tikz} % картинки в tikz
\usetikzlibrary{calc}
\usepackage{microtype} % свешивание пунктуации

\usepackage{floatrow} % для выравнивания рисунка и подписи
\usepackage{caption} % для пустых подписей

\usepackage{array} % для столбцов фиксированной ширины

\usepackage{indentfirst} % отступ в первом параграфе

\usepackage{sectsty} % для центрирования названий частей
\allsectionsfont{\centering}

\usepackage{amsmath, amsfonts} % куча стандартных математических плюшек

\usepackage{comment} % для комментариев

\usepackage{multicol} % текст в несколько колонок

\usepackage{lastpage} % чтобы узнать номер последней страницы

\usepackage{enumitem} % дополнительные плюшки для списков
%  например \begin{enumerate}[resume] позволяет продолжить нумерацию в новом списке

\usepackage{booktabs}

\usepackage{url} % для вставки интернет-ссылок

\usepackage{fontspec}
\usepackage{polyglossia}

\setmainlanguage{russian}
\setotherlanguages{english}

% download "Linux Libertine" fonts:
% http://www.linuxlibertine.org/index.php?id=91&L=1
\setmainfont{Linux Libertine O} % or Helvetica, Arial, Cambria
% why do we need \newfontfamily:
% http://tex.stackexchange.com/questions/91507/
\newfontfamily{\cyrillicfonttt}{Linux Libertine O}

\AddEnumerateCounter{\asbuk}{\russian@alph}{щ} % для списков с русскими буквами
\setlist[enumerate, 2]{label=\asbuk*),ref=\asbuk*}

\DeclareMathOperator{\Var}{Var}
\DeclareMathOperator{\E}{\mathbb{E}}

\let\P\relax
\DeclareMathOperator{\P}{\mathbb{P}}
\def\cN{\mathcal{N}}

\usepackage{fancyhdr} % весёлые колонтитулы
\pagestyle{fancy}
\lhead{Доказательства без слов}
\chead{}
\rhead{ЗПШ-2018}
\lfoot{}
\cfoot{}
\rfoot{\thepage/\pageref{LastPage}}
\renewcommand{\headrulewidth}{0.4pt}
\renewcommand{\footrulewidth}{0.4pt}


\begin{document}

\section{Анонс}

Доказательства часто бывают сложными и запутанными. А мы попытаемся увидеть самые очевидные доказательства — доказательства без слов :) Их мало в океане доказательств, но это настоящие жемчужинки!!! 

Почему площадь круга — это пи на радиус в квадрате? Как сложить натуральные числа от одного до ста? А как сложить их квадраты или кубы? Как сложить геометрическую прогрессию? Что такое треугольные числа? Как измерить площадь между следами колёс велосипеда? Кто такой Кавальери и почему он такой принципиальный? 

Мы попробуем увидеть ответы на эти и другие вопросы. Именно увидеть! В крайнем случае нащупать :)


\section{Презентация}

Презентация длится 10 минут, три дубля презентации для разных школьников. 
Объявленная аудитория 6-8 класс.

Простая загадка:
\[
   1 + 3 = ?^2
\]

Загадка посложнее:
\[
 1 + 3 + 5 + 7 + 9 + 11 = ?^2
\]

Геометрическое решение с помощью сложения уголков.

Пробуем 

\[
 1 + 3 + 5 + 7 + 9 + 11 + 13 + \ldots + 103 = ?^2
\]

Наводящие вопросы с надписями на доске:
\begin{enumerate}
  \item Сколько чисел от 1 до 100?
  \item Сколько нечётных чисел от 1 до 100?
  \item Сколько чисел в нужной нам сумме?
\end{enumerate}


Снова простая загадка:
\[
  \frac{1}{2} + \frac{1}{4} = ?
\]


Более сложная загадка:
\[
  \frac{1}{2} + \frac{1}{4} + \frac{1}{8} + \frac{1}{16} + \ldots = ?
\]


Рисуем квадрат и решаем геометрически. 
Здесь я делил квадрат на прямоугольные части. 
Полагаю, с отрезком или делением квадрата на треугольники должно быть менее понятно. 
Части отрезка чуть дальше подписываются,
а треугольники — сложнее прямоугольников.

А что будет с делением на три?

\[
  \frac{1}{3} + \frac{1}{9} + \frac{1}{27} + \frac{1}{81} + \ldots = ?
\]


\section{Встреча 1}


Новоселова Анастасия 7 \\
Бондарь Даниил 6 \\
Мещанинов Даниил 6 \\
Панасенков Виталий 6 \\
Платонов Степан 6 \\
Чайников Сергей 6 \\
Шевелев Валентин 6 \\
Сутормина Дарья 8 \\

\newpage
\begin{enumerate}

  \item Сколько здесь чисел: 
    \[
    7, 8, 9, \ldots, 48, 49, 50?
  \]
  \item Сколько здесь чисел: 
    \[
      11, 13, 15, \ldots, 53, 55, 57?
    \]
  \item Чему равна сумма: 
    \[
    1 + 2 + 3 + \ldots + 98 + 99 + 100?
    \]
  \item Миша и Петя подтягиваются «в лесенку». Сначала каждый подтягивается 1 раз, потом каждый 2 раза, потом 3 раза, и так далее до 10 раз. Потом они «спускаются» обратно до 1 подтягивания. Сколько раз каждый из них подтянулся, если Миша прошёл всю «лесенку» туда и обратно, а Петя — только туда до 10?
  \item 
 Чему равна сумма: 
    \[
    100 + 102 + 104 + \ldots + 198 + 200 + 202?
    \]
  \item Сколько на рисунке маленьких треугольничков?


\newcommand*\rows{10}
\begin{tikzpicture}[scale=0.5]
    \foreach \row in {0, 1, ...,\rows} {
        \draw ($\row*(0.5, {0.5*sqrt(3)})$) -- ($(\rows,0)+\row*(-0.5, {0.5*sqrt(3)})$);
        \draw ($\row*(1, 0)$) -- ($(\rows/2,{\rows/2*sqrt(3)})+\row*(0.5,{-0.5*sqrt(3)})$);
        \draw ($\row*(1, 0)$) -- ($(0,0)+\row*(0.5,{0.5*sqrt(3)})$);
    }
\end{tikzpicture}

  \item Сколько на рисунке вершинок?



\end{enumerate}

\newpage
\section{Встреча 2}

\begin{enumerate}

  \item Сколько примерно раз верёвочка равная радиусу окружности уложиться в окружность?

   Здесь все наугад пробовали ответить 4-5.
   Здесь я провёл эксперимент с удлинителем и нарисованной на доске окружностью. 
   Хорошо бы, чтобы каждый провёл его сам. 
   Наверное, нужна тонкая верёвочка и заготовки довольно крупных окружностей.

  \item Что такое $\pi$? Чему примерно равно $\pi$?

  \item Возьмём проволку, обмотаем её плотно вокруг апельсина. А затем увеличим радиус кольца из проволоки на 1 метр. Расположим апельсин так, чтобы зазор между ним и проволкой был везде одинаковым. Сможет ли в него пролезть кот Васька?

  \item Возьмём проволку, обмотаем её плотно вокруг земного шара. А затем увеличим радиус колцьа из проволоки на 1 метр. Расположим земной шар так, чтобы зазор между ним и проволкой был везде одинаковым. Сможет ли в него пролезть кот Васька? А хотя бы вирус Василий (он в 10000 раз меньше миллиметра)? Периметр Земли примерно равен 40000 километров.

    \item Нарисован прямоугольник размером $2\times 3$ клетки. Какой у него периметр? Какая площадь? Как изменится площадь и периметр, если все стороны увеличить в 2 раза?


  \item Отличница Машенька поставила кляксу. Периметр кляксы 1 см, а площадь 0.5 квадратных сантиметра. А двоечник Вовочка поставил кляксу такой же формы, только периметром 4 см. Какая у него площадь кляксы?

  \item В созвездии Малой Медведицы водятся Гигантомедведи. Гигантомедведь ничем не отличается от бурого земного, кроме того, что он выше его в 10 раз. Бурый медведь весит 500 кг, а в день съедает около 20 кг еды. Сколько весит Гигантомедведь? Сколько медведей в день может съесть Гигантомедведь?

  \item Найдите площадь маленького треугольника, если площадь большого равна 1:
    
    \vspace{5cm}


    % Найдите площадь например, треугольника, если по картинке видно во сколько раз один больше другого и известна площадь большого;

\end{enumerate}

\section{День 3}

\begin{enumerate}

  \item Повтор упражения на площадь маленькой и большой фигуры!!!

    Какая-нибудь смешная плоская фигура типа чайника!

  \item Найдите площадь закрашенного кусочка 
    % (закрашена одна какая-нибудь от подобной фигуры с известной площадью)


%Делим квадрат на 5 равных кусков и на 4 равных и один отличающийся

% Аналогично с шестиугольником

    \vspace{5cm}


\newpage
\item Запишите с помощью бесконечной суммы все  закрашенные площади 

 \vspace{5cm}
%\[
%  \frac{1}{4} + \frac{1}{4^2} + \frac{1}{4^3} + \frac{1}{4^4} + \ldots
%\]


\item Найдите сумму:
\[
  \frac{1}{4} + \frac{1}{4^2} + \frac{1}{4^3} + \frac{1}{4^4} + \ldots
\]


\item Запишите с помощью бесконечной суммы все  закрашенные площади 

  \vspace{5cm}
  
%(тут вращается кусочек в подобных шестиугольниках)
%\[
%  \frac{1}{7} + \frac{1}{7^2} + \frac{1}{7^3} + \frac{1}{7^4} + \ldots
%\]


\item Найдите сумму:
\[
  \frac{1}{7} + \frac{1}{7^2} + \frac{1}{7^3} + \frac{1}{7^4} + \ldots
\]
\end{enumerate}

\end{document}
