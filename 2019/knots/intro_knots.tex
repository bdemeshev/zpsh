\documentclass[12pt]{article}

\usepackage{hyperref} % гиперссылки

\usepackage{tikz} % картинки в tikz
\usetikzlibrary{arrows.meta} % tikz-прибамбас для рисовки стрелочек подлиннее

\usepackage{microtype} % свешивание пунктуации

\usepackage{array} % для столбцов фиксированной ширины

\usepackage{indentfirst} % отступ в первом параграфе

\usepackage{sectsty} % для центрирования названий частей
\allsectionsfont{\centering}

\usepackage{amsmath} % куча стандартных математических плюшек
\usepackage{amssymb} % символы
\usepackage{amsthm} % теоремки

\usepackage{comment} % добавление длинных комментариев

\usepackage[top=2cm, left=1.2cm, right=1.2cm, bottom=2cm]{geometry} % размер текста на странице

\usepackage{lastpage} % чтобы узнать номер последней страницы

\usepackage{enumitem} % дополнительные плюшки для списков
%  например \begin{enumerate}[resume] позволяет продолжить нумерацию в новом списке

\usepackage{caption} % что-то делает с подписями рисунков :)

\usepackage{qcircuit} % для рисовки квантовых диаграмм
\usepackage{physics} % бракеты

\usepackage{answers} % разделение условий и ответов в упражнениях


\usepackage{fancyhdr} % весёлые колонтитулы
\pagestyle{fancy}
\lhead{Квантовые вычисления}
\chead{}
\rhead{КЛШ-2018}
\lfoot{}
\cfoot{}
\rfoot{\thepage/\pageref{LastPage}}
\renewcommand{\headrulewidth}{0.4pt}
\renewcommand{\footrulewidth}{0.4pt}



\usepackage{todonotes} % для вставки в документ заметок о том, что осталось сделать
% \todo{Здесь надо коэффициенты исправить}
% \missingfigure{Здесь будет Последний день Помпеи}
% \listoftodos — печатает все поставленные \todo'шки



\usepackage{booktabs} % красивые таблицы
% заповеди из докупентации:
% 1. Не используйте вертикальные линни
% 2. Не используйте двойные линии
% 3. Единицы измерения - в шапку таблицы
% 4. Не сокращайте .1 вместо 0.1
% 5. Повторяющееся значение повторяйте, а не говорите "то же"



\usepackage{fontspec} % что-то про шрифты?
\usepackage{polyglossia} % русификация xelatex

\setmainlanguage{russian}
\setotherlanguages{english}

% download "Linux Libertine" fonts:
% http://www.linuxlibertine.org/index.php?id=91&L=1
\setmainfont{Linux Libertine O} % or Helvetica, Arial, Cambria
% why do we need \newfontfamily:
% http://tex.stackexchange.com/questions/91507/
\newfontfamily{\cyrillicfonttt}{Linux Libertine O}

\AddEnumerateCounter{\asbuk}{\russian@alph}{щ} % для списков с русскими буквами
\setlist[enumerate, 2]{label=\asbuk*),ref=\asbuk*}

%% эконометрические сокращения
\DeclareMathOperator{\Cov}{Cov}
\DeclareMathOperator{\Arg}{Arg}
\DeclareMathOperator{\Corr}{Corr}
\DeclareMathOperator{\Var}{Var}
\DeclareMathOperator{\E}{\mathbb{E}}
\def \hb{\hat{\beta}}
\def \hs{\hat{\sigma}}
\def \htheta{\hat{\theta}}
\def \s{\sigma}
\def \hy{\hat{y}}
\def \hY{\hat{Y}}
\def \v1{\vec{1}}
\def \e{\varepsilon}
\def \he{\hat{\e}}
\def \z{z}
\def \hVar{\widehat{\Var}}
\def \hCorr{\widehat{\Corr}}
\def \hCov{\widehat{\Cov}}
\def \cN{\mathcal{N}}
\let\P\relax
\DeclareMathOperator{\P}{\mathbb{P}}



\usepackage[bibencoding = auto,
backend = biber,
sorting = none,
style=alphabetic]{biblatex}

\addbibresource{em1_pset_v2.bib}



% делаем короче интервал в списках
\setlength{\itemsep}{0pt}
\setlength{\parskip}{0pt}
\setlength{\parsep}{0pt}




\Newassociation{sol}{solution}{solution_file}
% sol --- имя окружения внутри задач
% solution --- имя окружения внутри solution_file
% solution_file --- имя файла в который будет идти запись решений
% можно изменить далее по ходу
\Opensolutionfile{solution_file}[all_solutions]
% в квадратных скобках фактическое имя файла

% магия для автоматических гиперссылок задача-решение
\newlist{myenum}{enumerate}{3}
% \newcounter{problem}[chapter] % нумерация задач внутри глав
\newcounter{problem}[section]

\newenvironment{problem}%
{%
\refstepcounter{problem}%
%  hyperlink to solution
     \hypertarget{problem:{\thesection.\theproblem}}{} % нумерация внутри глав
     % \hypertarget{problem:{\theproblem}}{}
     \Writetofile{solution_file}{\protect\hypertarget{soln:\thesection.\theproblem}{}}
     %\Writetofile{solution_file}{\protect\hypertarget{soln:\theproblem}{}}
     \begin{myenum}[label=\bfseries\protect\hyperlink{soln:\thesection.\theproblem}{\thesection.\theproblem},ref=\thesection.\theproblem]
     % \begin{myenum}[label=\bfseries\protect\hyperlink{soln:\theproblem}{\theproblem},ref=\theproblem]
     \item%
    }%
    {%
    \end{myenum}}
% для гиперссылок обратно надо переопределять окружение
% это происходит непосредственно перед подключением файла с решениями



\theoremstyle{definition}
\newtheorem{definition}{Определение}



\begin{document}

\tableofcontents{}

\section*{Цель}

\section{Презентация}



\section{Встреча 1}

На занятие нужны: верёвки ~1.5 метра на каждый нос, можно длиннее,
и кусочки скотча для склеивания
верёвки в кольцо.


Топология — наука про плавную математику и непрерывные деформации.

Например, топология изучает амёб. Амёбы могут отращивать отростки, сжимать
свою часть, растягивать свою часть. Не могут схлопывать пузырьки воздуха вокруг себя,
не могут соединять свои части, не могут разрывать свои части.

Может ли запутанная амёба с двумя связанными рожками распутаться?
Может ли амёба, запутавшаяся на фитнесе, снять одну ногу с обруча? А вторую?
Про вторую пообещал доказать.

Мы не будем изучать амёб, а будем изучать узлы. Чем узел отличается от амёбы?
Не имеет толщины, не можем отращивать отростки, нет раздвоения на две части.

Узел — замкнутая ломаная из нескольких звеньев в 3D без самопересечений.
Ориентированный узел — задано направление. Как задать направление?
Школьники ответили стрелочкой?

Упражнение: нарисуйте диаграмму узла.
Требование: в одной точке на диаграмме могут только две линии пересекаться.
Линия не ложится на линию.

Записываем узел. Метод муравьишки. Буквы, чтобы назвать точки пересечения,
сверху или снизу, направление обхода.

Упражнение: свяжите несложный узел (до 5 пересечений). Нарисуйте. Запишите методом
муравьишки.

Узлы эквивалентны, если один из них можно перевести в другой плавными движениями
в 3D. Какие действия можно делать на диаграмме?

Движения Рейдеместера.

Здесь мы попробовали распутать узел с помощью движений Рейдеместера и школьники хотели
делать действие, которое разрешено, но не является Рейдеместером.
Нужно объяснить, как меняется количество пересечений при действии Рейдеместера (!).

Вяжем булинь. Ему 3000 лет, ещё строители пирамид его вязали.


\section{Встреча 2}

Повторили мысль: человеку очевидно больше движений, чем движения Рейдеместера.
Но они прекрасны тем, что их хватает, чтобы превратить узел в любой эквивалентный ему.

Упражнение:

Сколько разных узлов на листочке с 9 узлами? Соедините равные узлы.
Подсказка: на бумаге тяжело решать. Свяжите первый и пробуйте его перевести в остальные.


Инвариант. Здесь хорошо бы привести пример красивой простой задачки на инвариант.


Узел — количество правильных раскрасок.

Раскраска в 3 цвета называется правильной, если
- всего использовано не больше трёх цветов
- в пересечении сходятся 1 или 3 цвета, но не два.

Сколько раскрасок у тривиального узла? У трилистника?

Повторяем движения Рейдеместера.
Доказательство инвариантности числа правильных раскрасок.


Восьмёрка. Восьмёрка сложенным вдвое концом.
Различают ли раскраски восьёрку и тривиальный узел? (не успели спросить!)


Повторяем булинь. Довязываем контрольный узелок на булинь.
Вяжем голландский морской булинь.

\section{Встреча 3}


Зацепление — несколько узлов.
Зацепление Хопфа и тривиальное зацепление.

Число раскрасок: повторяем, что оно не различает восьмёрку и тривиальный узел.

Повторяем график линейной функции и параболы по точкам.

Многочлен Конвея.

Многочлен Конвея различает даже правое и левое зацепление Хопфа!

Вязать узлы (кроме восьмёрки) не успели.

\section{Встреча 4}

Вспоминаем многочлены Конвея посчитанные вчера.
Многочлен Конвея для восьмёрки. Итого, многочлен Конвея различает
трилистник и восьмёрку.

Как повесить картину на два гвоздя?

Дал определение мега-гвоздя и обход мега-гвоздя, заменяющий обход двух гвоздей
Повесили картину на три гвоздя, но семиклассникам было тяжело. Миша живо следил,
остальным былы тяжко.
Хотя это мог бы быть эффектный трюк для проектов.

Успели шкотовый узел (через оттопыренный большой палец).
Брам-шкотовый не успели.

\section{Встреча 5}

Вязали шкотовый и брам-шкотовый узлы.

Сумма узлов. (лучше было убрать и больше времени посвятить косам)

Вопрос "Правда ли, что результат не зависит от места склейки?"
Даня наобум ответил нет, Ваня думал и сначала ответил да, но потом
Ваня довольно внятно изложил технику «протащи узелок сквозь узелок».

Косы. Примеры кос.

Теорема Александера (формулировка)

Решили пример: по косе нарисуйте узел. Пример по узлу нарисуйте косу не успели.

Записали пару кос формулами через базовые косы.

Соотношения между базовыми косами начали, но не успели.


\Closesolutionfile{solution_file}

% для гиперссылок на условия
% http://tex.stackexchange.com/questions/45415
\renewenvironment{solution}[1]{%
         % add some glue
         \vskip .5cm plus 2cm minus 0.1cm%
         {\bfseries \hyperlink{problem:#1}{#1.}}%
}%
{%
}%

\section{Решения}
\input{all_solutions}


\section{Источники мудрости}
\printbibliography[heading=none]


\end{document}
