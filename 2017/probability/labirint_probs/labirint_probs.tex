\documentclass[a4paper, 12pt]{article}


\usepackage{mathrsfs}

\usepackage{amscd}
\usepackage[paper=a4paper,
top=2cm, bottom=2cm, left=1cm, right=1cm, includefoot]{geometry} % размер листа бумаги


\usepackage{tikz} % картинки в tikz
\usepackage{microtype} % свешивание пунктуации

\usepackage{floatrow} % для выравнивания рисунка и подписи
\usepackage{caption} % для пустых подписей

\usepackage{array} % для столбцов фиксированной ширины

\usepackage{indentfirst} % отступ в первом параграфе

\usepackage{sectsty} % для центрирования названий частей
\allsectionsfont{\centering}

\usepackage{amsmath, amsfonts} % куча стандартных математических плюшек

\usepackage{comment} % для комментариев

\usepackage{multicol} % текст в несколько колонок

\usepackage{lastpage} % чтобы узнать номер последней страницы

\usepackage{enumitem} % дополнительные плюшки для списков
%  например \begin{enumerate}[resume] позволяет продолжить нумерацию в новом списке

\usepackage{booktabs}

\usepackage{url} % для вставки интернет-ссылок

\usepackage{fontspec}
\usepackage{polyglossia}

\setmainlanguage{russian}
\setotherlanguages{english}

% download "Linux Libertine" fonts:
% http://www.linuxlibertine.org/index.php?id=91&L=1
\setmainfont{Linux Libertine O} % or Helvetica, Arial, Cambria
% why do we need \newfontfamily:
% http://tex.stackexchange.com/questions/91507/
\newfontfamily{\cyrillicfonttt}{Linux Libertine O}

\AddEnumerateCounter{\asbuk}{\russian@alph}{щ} % для списков с русскими буквами
\setlist[enumerate, 2]{label=\asbuk*),ref=\asbuk*}

\DeclareMathOperator{\Var}{Var}
\DeclareMathOperator{\E}{\mathbb{E}}

\let\P\relax
\DeclareMathOperator{\P}{\mathbb{P}}
\def\cN{\mathcal{N}}

\usepackage{fancyhdr} % весёлые колонтитулы
\pagestyle{fancy}
\lhead{Теория вероятностей}
\chead{}
\rhead{ЗПШ-2017}
\lfoot{}
\cfoot{}
\rfoot{\thepage/\pageref{LastPage}}
\renewcommand{\headrulewidth}{0.4pt}
\renewcommand{\footrulewidth}{0.4pt}


\begin{document}

% источник задач:
% * группа vk Синяя птица (вероятность в школе)
% Шень
% Канеман

\begin{enumerate}
\item (1б) Петя и Вася одновременно подбрасывают по одному игральному кубику. У кого выпало больше, тот и выиграл. Если выпало поровну, то объявляется ничья. Какова вероятность того, что будет ничья?
\item (1б) На ёлку пришли: Вася с братом Мишей, Аня с сестрой Настей, Петя с братом Костей и Катя с братом Васей. Дед Мороз устроил лотерею. Какова вероятность того, что главный приз достался мальчику?
\item (1б) Игральный кубик подбрасывают два раза. Сколько в среднем очков выпадает в сумме на двух кубиках?
\item (1б) В игре бросают кубик; выигрышем считается выпадение пятёрки или шестёрки. Сколько (примерно) выигрышей будет в длинной серии из 666 игр?
\item (1б) В мешке лежат бумажки с цифрами 1, 2, \ldots ,9. Из мешка наудачу вытаскивают одну из бумажек наугад. Какова вероятность того, что будет вытащено чётное число?
\item (1б) В мешке лежат бумажки с цифрами 1, 2, \ldots, 9. Из мешка наудачу вытаскивают одну из бумажек наугад. Какова вероятность того, что будет вытащено число, делящееся на 3?
\item (1б) В мешке лежат бумажки с цифрами 1, 2, \ldots, 9. Из мешка наудачу вытаскивают одну из бумажек наугад. Какова вероятность того, что будет вытащено число, не делящееся на 3?
\item (1б) В мешке лежат бумажки с цифрами 1, 2, \ldots, 9. Из мешка наудачу вытаскивают одну из бумажек наугад. Какова вероятность того, что будет вытащено число, не делящееся ни на 2, ни на 3?
\item (1б) Маша идёт на день рождения, где будут десять мальчиков и десять девочек (кроме Маши). Они садятся за круглый стол в случайном порядке. Какова вероятность, что справа от Маши будет сидеть мальчик?
\item (1б) Маша идёт на день рождения, где будут пять мальчиков и пять девочек (кроме Маши). Они садятся за круглый стол в случайном порядке. Какова вероятность, что справа от Маши будет сидеть мальчик?
\item (1б) Обычную рублевую монетку подбрасывают четыре раза. Первые три раза она выпала орлом. Какова вероятность того, что она выпадет орлом в четвертый раз?
\item (1б) У Пети связка из 10 ключей. Один из них подходит к замку. Петя не знает, какой ключ подходит к замку и перебирает их по очереди. У какого ключа выше шансы подойти?
\item (1б) Одно из десяти чисел увеличили на 1. Как изменилось от этого среднее арифметическое этих чисел?
\item (1б) В автобусе собралась футбольная команда из 11 человек и волейбольная команда из 6 человек. Средний возраст футболистов — 37 лет, средний возраст волейболистов — 20 год. Каков средний возраст пассажиров автобуса?
\end{enumerate}

\newpage
\begin{enumerate}
  \item (2б) Студент Мгамба Унь сдаёт экзамен по английском языку в России. Ему дали 5 карточек с английскими словами и 5 карточек с их переводами на русский. Мгамба не знает ни английского, ни русского и сопоставляет карточки наугад. Какова вероятность того, что он угадает все переводы?
  \item (2б) На ёлку пришли: Вася с братом Мишей, Аня с сестрой Настей, Петя с братом Костей и Катя с братом Васей. Дед Мороз устроил лотерею и главный приз достался мальчику. Какова вероятность того, что тот пришёл с братом?
  \item (2б) В классе 20 человек. Один заболел, ещё один — опоздал. Какова вероятность того, что их фамилии идут подряд в классном журнале?
  \item (2б) Вероятность рождения двойняшек в Урляндии равна $1/10$. А тройняшки и больше чем тройняшки в Урляндии не рождаются. Какова вероятность того, что первый встречный урляндец один из двойняшек?
  \item (2б) Петя и Вася играют в дурака до 6 побед. Сейчас счёт 5:4 в пользу Пети. На кону 36 рублей, и тут внезапно начался ураган. Петя и Вася вынуждены прервать игры. Как им поделить деньги по справедливости?
  \item (2б) Петя и Вася играют в дурака до 4 побед. Сейчас счёт 3:2 в пользу Пети. На кону 20 рублей, и тут внезапно начался ураган. Петя и Вася вынуждены прервать игры. Как им поделить деньги по справедливости?
  \item (2б) Вася подбрасывает два кубика. Какая сумма очков на кубиках наиболее вероятна?
  \item (2б) Маша переставляет буквы в слове МАША в случайном порядке. Какова вероятность того, что снова получится слово МАША?
  \item (2б) Маша переставляет буквы в слове МАМА в случайном порядке. Какова вероятность того, что снова получится слово МАМА?
  \item (2б) Среди учеников школы 15\% знают французский язык и 20\% знают немецкий язык. Доля учеников, знающих оба этих языка, составляет 5\%. Какова доля учеников, знающих французский язык, среди учеников, знающих немецкий язык?
  \item (2б) Среди учеников школы 15\% знают французский язык и 20\% знают немецкий язык. Доля учеников, знающих оба этих языка, составляет 5\%. Какова доля учеников, знающих французский язык, среди учеников, не знающих немецкий язык?
  \item (2б) Среди учеников школы 15\% знают французский язык и 20\% знают немецкий язык. Доля учеников, знающих оба этих языка, составляет 5\%. Какова доля учеников, знающих немецкий язык, среди учеников, знающих французский язык?
  \item (2б) Среди учеников школы 15\% знают французский язык и 20\% знают немецкий язык. Доля учеников, знающих оба этих языка, составляет 5\%. Какова доля учеников, знающих немецкий язык, среди учеников, не знающих французский язык?
  \item (2б) Среди шахматистов каждый седьмой — музыкант, а среди музыкантов каждый девятый — шахматист. Кого больше, шахматистов или музыкантов и во сколько раз?
\end{enumerate}

\newpage
\begin{enumerate}
\item (3б) В классе не более 40 человек, среди них есть те, кого зовут Коля. Вероятность того, что случайно выбранный ученик выше всех Коль, равна $2/5$. Вероятность того, что случайно выбранный ученик класса ниже всех Коль, равна $3/7$. Сколько Коль может быть в классе?
\item (3б) Джон Сильвер и Билли Бонс играют в кости. У них есть одна игральная кость и они по очереди её бросают. Кто первый выбросит шестёрку, тот и выиграл. Начинает Джон Сильвер. Какова вероятность того, что победит Билли Бонс?
\item (3б) Ровно половина жителей острова Невезения зайцы, а остальные — кролики. Зайцы врут в половине своих фраз, а кролики — в двух третях. Вышел однажды житель острова, сел на пенёк и сказал: «Я не заяц». Какова условная вероятность того, что он действительно не заяц?
\item (3б) Ровно половина жителей острова Невезения зайцы, а остальные — кролики. Зайцы врут в половине своих фраз, а кролики — в двух третях. Вышел однажды житель острова, сел на пенёк и сказал: «Я заяц». Какова условная вероятность того, что он действительно заяц?
\item (3б) Ровно половина жителей острова Невезения зайцы, а остальные — кролики. Зайцы врут в половине своих фраз, а кролики — в двух третях. Вышел однажды житель острова, сел на пенёк и сказал: «Я кролик». Какова условная вероятность того, что он действительно кролик?
\item (3б) Ровно половина жителей острова Невезения зайцы, а остальные — кролики. Зайцы врут в половине своих фраз, а кролики — в двух третях. Вышел однажды житель острова, сел на пенёк и сказал: «Я не кролик». Какова условная вероятность того, что он действительно не кролик?
%\item (3б) Ровно половина жителей острова Невезения зайцы, а остальные — кролики. Зайцы врут в половине своих фраз, а кролики — в двух третях. Вышел однажды житель острова, сел на пенёк и сказал: «Я не заяц». А потом помолчал и добавил: «Я не кролик». Какова условная вероятность того, что он всё же заяц?
\item (3б) Джон Сильвер подбрасывает игральную кость до тех пор, пока не выпадет 6. Сколько в среднем бросков ему потребуется?
\item (3б) Джон Сильвер подбрасывает игральную кость до тех пор, пока не выпадет 6 два раза, не обязательно подряд. Сколько в среднем бросков ему потребуется?
\item (3б) В пакетике 6 оранжевых и $n$ жёлтых конфет. Аня достаёт одну наугад и съедает. Затем достаёт ещё одну и снова съедает. Вероятность того, что Аня съела две оранжевых равна $1/3$. Сколько конфет было в пакетике?
\item (3б) Редкой болезнью болеет 1\% населения. Существующий тест ошибается в 10\% случаев. У первого встречного берут тест. Судя по тесту, человек болен. Какова вероятность того, что он действительно болен?
\item (3б) У Паши 4 ореха. Из них два, не ясно какие, пустые. Паша разбивает первый орех, и затем, не глядя на результат, разбивает второй. Второй разбитый орех — пустой. Вероятность того, что первый разбитый орех был пустым?
\end{enumerate}

\newpage
Ответы
\begin{enumerate}
\item (1б) Петя и Вася одновременно подбрасывают по одному игральному кубику. У кого выпало больше, тот и выиграл. Если выпало поровну, то объявляется ничья. Какова вероятность того, что будет ничья? $1/6$
\item (1б) На ёлку пришли: Вася с братом Мишей, Аня с сестрой Настей, Петя с братом Костей и Катя с братом Васей. Дед Мороз устроил лотерею. Какова вероятность того, что главный приз достался мальчику? $5/8$
\item (1б) Игральный кубик подбрасывают два раза. Сколько в среднем очков выпадает в сумме на двух кубиках? $7$
\item (1б) В игре бросают кубик; выигрышем считается выпадение пятёрки или шестёрки. Сколько (примерно) выигрышей будет в длинной серии из 666 игр? $666/3=222$
\item (1б) В мешке лежат бумажки с цифрами 1, 2, \ldots ,9. Из мешка наудачу вытаскивают одну из бумажек наугад. Какова вероятность того, что будет вытащено чётное число? $4/9$
\item (1б) В мешке лежат бумажки с цифрами 1, 2, \ldots, 9. Из мешка наудачу вытаскивают одну из бумажек наугад. Какова вероятность того, что будет вытащено число, делящееся на 3? $3/9=1/3$
\item (1б) В мешке лежат бумажки с цифрами 1, 2, \ldots, 9. Из мешка наудачу вытаскивают одну из бумажек наугад. Какова вероятность того, что будет вытащено число, не делящееся на 3? $6/9=2/3$
\item (1б) В мешке лежат бумажки с цифрами 1, 2, \ldots, 9. Из мешка наудачу вытаскивают одну из бумажек наугад. Какова вероятность того, что будет вытащено число, не делящееся ни на 2, ни на 3? $3/9=1/3$
\item (1б) Маша идёт на день рождения, где будут десять мальчиков и десять девочек (кроме Маши). Они садятся за круглый стол в случайном порядке. Какова вероятность, что справа от Маши будет сидеть мальчик? $10/20=1/2$
\item (1б) Маша идёт на день рождения, где будут пять мальчиков и пять девочек (кроме Маши). Они садятся за круглый стол в случайном порядке. Какова вероятность, что справа от Маши будет сидеть мальчик? $5/10=1/2$
\item (1б) Обычную рублевую монетку подбрасывают четыре раза. Первые три раза она выпала орлом. Какова вероятность того, что она выпадет орлом в четвертый раз? $1/2$
\item (1б) У Пети связка из 10 ключей. Один из них подходит к замку. Петя не знает, какой ключ подходит к замку и перебирает их по очереди. У какого ключа выше шансы подойти? У всех одинаковы
\item (1б) Одно из десяти чисел увеличили на 1. Как изменилось от этого среднее арифметическое этих чисел? На $0.1$
\item (1б) В автобусе собралась футбольная команда из 11 человек и волейбольная команда из 6 человек. Средний возраст футболистов — 37 лет, средний возраст волейболистов — 20 год. Каков средний возраст пассажиров автобуса? $(37 \cdot 11 + 6 \cdot 20)/17=31$
\end{enumerate}

\newpage
\begin{enumerate}
  \item (2б) Студент Мгамба Унь сдаёт экзамен по английском языку в России. Ему дали 5 карточек с английскими словами и 5 карточек с их переводами на русский. Мгамба не знает ни английского, ни русского и сопоставляет карточки наугад. Какова вероятность того, что он угадает все переводы? $1/5!=1/120$.
  \item (2б) На ёлку пришли: Вася с братом Мишей, Аня с сестрой Настей, Петя с братом Костей и Катя с братом Васей. Дед Мороз устроил лотерею и главный приз достался мальчику. Какова вероятность того, что тот пришёл с братом? $4/5$.
  \item (2б) В классе 20 человек. Один заболел, ещё один — опоздал. Какова вероятность того, что их фамилии идут подряд в классном журнале? $19/C_{20}^2=0.1$
  \item (2б) Вероятность рождения двойняшек в Урляндии равна $1/10$. А тройняшки и больше чем тройняшки в Урляндии не рождаются. Какова вероятность того, что первый встречный урляндец один из двойняшек? $2/11$.
  \item (2б) Петя и Вася играют в дурака до 6 побед. Сейчас счёт 5:4 в пользу Пети. На кону 36 рублей, и тут внезапно начался ураган. Петя и Вася вынуждены прервать игры. Как им поделить деньги по справедливости?

  В пропорции $3:1$, то есть $27:9$.

  \item (2б) Петя и Вася играют в дурака до 4 побед. Сейчас счёт 3:2 в пользу Пети. На кону 20 рублей, и тут внезапно начался ураган. Петя и Вася вынуждены прервать игры. Как им поделить деньги по справедливости?

  В пропорции $3:1$, то есть $15:5$.
  \item (2б) Вася подбрасывает два кубика. Какая сумма очков на кубиках наиболее вероятна? $7$.
  \item (2б) Маша переставляет буквы в слове МАША в случайном порядке. Какова вероятность того, что снова получится слово МАША? $2/4!=1/12$
  \item (2б) Маша переставляет буквы в слове МАМА в случайном порядке. Какова вероятность того, что снова получится слово МАМА? $4/4!=1/6$
  \item (2б) Среди учеников школы 15\% знают французский язык и 20\% знают немецкий язык. Доля учеников, знающих оба этих языка, составляет 5\%. Какова доля учеников, знающих французский язык, среди учеников, знающих немецкий язык? $5/20=1/4$
  \item (2б) Среди учеников школы 15\% знают французский язык и 20\% знают немецкий язык. Доля учеников, знающих оба этих языка, составляет 5\%. Какова доля учеников, знающих французский язык, среди учеников, не знающих немецкий язык? $10/80=1/8$
  \item (2б) Среди учеников школы 15\% знают французский язык и 20\% знают немецкий язык. Доля учеников, знающих оба этих языка, составляет 5\%. Какова доля учеников, знающих немецкий язык, среди учеников, знающих французский язык? $5/15=1/3$
  \item (2б) Среди учеников школы 15\% знают французский язык и 20\% знают немецкий язык. Доля учеников, знающих оба этих языка, составляет 5\%. Какова доля учеников, знающих немецкий язык, среди учеников, не знающих французский язык? $15/85=3/17$
  \item (2б) Среди шахматистов каждый седьмой — музыкант, а среди музыкантов каждый девятый — шахматист. Кого больше, шахматистов или музыкантов и во сколько раз? Музыкантов в $9/7$
\end{enumerate}

\newpage
\begin{enumerate}
\item (3б) В классе не более 40 человек, среди них есть те, кого зовут Коля. Вероятность того, что случайно выбранный ученик выше всех Коль, равна $2/5$. Вероятность того, что случайно выбранный ученик класса ниже всех Коль, равна $3/7$. Сколько Коль может быть в классе?

В классе $5\cdot 7=35$ человек. Выше всех Коль 14 человек, ниже всех Коль 15 человек. Значит Коль от 1 до 16.
\item (3б) Джон Сильвер и Билли Бонс играют в кости. У них есть одна игральная кость и они по очереди её бросают. Кто первый выбросит шестёрку, тот и выиграл. Начинает Джон Сильвер. Какова вероятность того, что победит Билли Бонс? $p=5/11$.
\item (3б) Ровно половина жителей острова Невезения зайцы, а остальные — кролики. Зайцы врут в половине своих фраз, а кролики — в двух третях. Вышел однажды житель острова, сел на пенёк и сказал: «Я не заяц». Какова условная вероятность того, что он действительно не заяц? $10/25=0.4$
\item (3б) Ровно половина жителей острова Невезения зайцы, а остальные — кролики. Зайцы врут в половине своих фраз, а кролики — в двух третях. Вышел однажды житель острова, сел на пенёк и сказал: «Я заяц». Какова условная вероятность того, что он действительно заяц? $15/35=3/7$
\item (3б) Ровно половина жителей острова Невезения зайцы, а остальные — кролики. Зайцы врут в половине своих фраз, а кролики — в двух третях. Вышел однажды житель острова, сел на пенёк и сказал: «Я кролик». Какова условная вероятность того, что он действительно кролик? $10/25=0.4$
\item (3б) Ровно половина жителей острова Невезения зайцы, а остальные — кролики. Зайцы врут в половине своих фраз, а кролики — в двух третях. Вышел однажды житель острова, сел на пенёк и сказал: «Я не кролик». Какова условная вероятность того, что он действительно не кролик? $15/35=3/7$
%\item (3б) Ровно половина жителей острова Невезения зайцы, а остальные — кролики. Зайцы врут в половине своих фраз, а кролики — в двух третях. Вышел однажды житель острова, сел на пенёк и сказал: «Я не заяц». А потом помолчал и добавил: «Я не кролик». Какова условная вероятность того, что он всё же заяц?
\item (3б) Джон Сильвер подбрасывает игральную кость до тех пор, пока не выпадет 6. Сколько в среднем бросков ему потребуется? $6$ бросков.
\item (3б) Джон Сильвер подбрасывает игральную кость до тех пор, пока не выпадет 6 два раза, не обязательно подряд. Сколько в среднем бросков ему потребуется? $6+6=12$.
\item (3б) В пакетике 6 оранжевых и $n$ жёлтых конфет. Аня достаёт одну наугад и съедает. Затем достаёт ещё одну и снова съедает. Вероятность того, что Аня съела две оранжевых равна $1/3$. Сколько конфет было в пакетике? $\frac{6}{6+n}\frac{5}{5+n}=\frac{1}{3}$, $6+n=10$.
\item (3б) Редкой болезнью болеет 1\% населения. Существующий тест ошибается в 10\% случаев. У первого встречного берут тест. Судя по тесту, человек болен. Какова вероятность того, что он действительно болен? $0.01\cdot 0.9/(0.01 \cdot 0.9 + 0.99 \cdot 0.1)=1/12$
\item (3б) У Паши 4 ореха. Из них два, не ясно какие, пустые. Паша разбивает первый орех, и затем, не глядя на результат, разбивает второй. Второй разбитый орех — пустой. Вероятность того, что первый разбитый орех был пустым? $1/3$.
\end{enumerate}


\end{document}
