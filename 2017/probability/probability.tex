\documentclass[a4paper, 12pt]{article}


\usepackage{mathrsfs}

\usepackage{amscd}
\usepackage[paper=a4paper,
top=2cm, bottom=2cm, left=2cm, right=2cm, includefoot]{geometry} % размер листа бумаги


\usepackage{tikz} % картинки в tikz
\usepackage{microtype} % свешивание пунктуации

\usepackage{floatrow} % для выравнивания рисунка и подписи
\usepackage{caption} % для пустых подписей

\usepackage{array} % для столбцов фиксированной ширины

\usepackage{indentfirst} % отступ в первом параграфе

\usepackage{sectsty} % для центрирования названий частей
\allsectionsfont{\centering}

\usepackage{amsmath, amsfonts} % куча стандартных математических плюшек

\usepackage{comment} % для комментариев

\usepackage{multicol} % текст в несколько колонок

\usepackage{lastpage} % чтобы узнать номер последней страницы

\usepackage{enumitem} % дополнительные плюшки для списков
%  например \begin{enumerate}[resume] позволяет продолжить нумерацию в новом списке

\usepackage{booktabs}

\usepackage{url} % для вставки интернет-ссылок

\usepackage{fontspec}
\usepackage{polyglossia}

\setmainlanguage{russian}
\setotherlanguages{english}

% download "Linux Libertine" fonts:
% http://www.linuxlibertine.org/index.php?id=91&L=1
\setmainfont{Linux Libertine O} % or Helvetica, Arial, Cambria
% why do we need \newfontfamily:
% http://tex.stackexchange.com/questions/91507/
\newfontfamily{\cyrillicfonttt}{Linux Libertine O}

\AddEnumerateCounter{\asbuk}{\russian@alph}{щ} % для списков с русскими буквами
\setlist[enumerate, 2]{label=\asbuk*),ref=\asbuk*}

\DeclareMathOperator{\Var}{Var}
\DeclareMathOperator{\E}{\mathbb{E}}

\let\P\relax
\DeclareMathOperator{\P}{\mathbb{P}}
\def\cN{\mathcal{N}}

\usepackage{fancyhdr} % весёлые колонтитулы
\pagestyle{fancy}
\lhead{Теория вероятностей}
\chead{}
\rhead{ЗПШ-2017}
\lfoot{}
\cfoot{}
\rfoot{\thepage/\pageref{LastPage}}
\renewcommand{\headrulewidth}{0.4pt}
\renewcommand{\footrulewidth}{0.4pt}


\begin{document}

\section{Встреча 1}

Пришло 9 школьников, 2 вожатых.

Техника решения задач и обозначения.

\subsection{Табличка}

Составили табличку 2 на 2: сколько 10 классников и не 10 классников, сколько любят ваниальное мороженое, сколько не любят. Предположим, что один из нас (из 12), мистер или мисс $X$, случайно заканчивает обедать первым.

\begin{tabular}{ccc}
\toprule
 & V+ & V- \\
\midrule
10+ & 6 & 1 \\
10- & 2 & 3 \\
\bottomrule
\end{tabular}

Какова вероятность, что это будет 10-классник? Любитель ванильного? Одновременно 10-классник И любитель ванильного?
10-классник ИЛИ любитель ванильного? 10-классник ЕСЛИ любитель ванильного?

Значки $\cap$, $\cup$, $|$.

\subsection{Дерево}

Красная Шапочка (КШ) выбирает тропинки наугад и равновероятно.

Сначала дорога делится на три: L, M и R. L делится на две, R делится на 3, R делится на две.

Дороги LL, LR, ML, MM, MR патрулирует Волк.

Дороги LL, LR, MR ведут в Овраг.

Дороги ML, MM, RL, RR ведут к Бабушке.


События:

\begin{itemize}
  \item Б — КШ попадёт к Бабушке
  \item В - КШ встретит Волка
\end{itemize}

$P(\text{Б})$? $P(\text{В})$? $P(\text{Б}\cap \text{В})$? $P(\text{Б} | \text{В})$?

\section{Встреча 2}

Пришло два новых школьника и одна вожатая.

Начали с повторения формулы
\[
P(A|B) = \frac{P(A\cap B)}{P(B)}.
\]

Рисую два пересекающихся круга $A$ и $B$ на доске. Вопрос: какова вероятность попадания тряпкой в круг $A$, если известно, что я попал в круг $B$? Для наглядности кидаю тряпку.

Даже слабые школьники сказали, что условная вероятность есть отношение площадей:
\[
P(A|B) = \frac{S(A\cap B)}{S(B)}.
\]

Так формулу условной вероятности и мотивировали.

Далее решили 1а, 1б, 1в, 1г самостоятельно, 2а, 2в, 2б. Задачу 2б решали после 2в, так как она требует дорисовки дерева. На дом задал задачу 2г.

\newpage
\begin{enumerate}

\item  В городе примерно 4\% такси зелёного цвета и остальные жёлтые. Свидетель путает цвет на показаниях в суде с вероятностью 10\%.

\begin{enumerate}
\item Какова вероятность того, свидетель скажет, что видел зелёное такси?
\item Какова вероятность того, свидетель ошибётся?
\item Какова вероятность того, что такси было зелёным, если свидетель говорит, что оно было зелёным?
\item Какова вероятность того, что такси было жёлтым, если свидетель говорит, что оно было жёлтым?
\end{enumerate}


\item У тети Маши — двое детей, один старше другого. Предположим, что вероятности рождения мальчика и девочки равны и не зависят от дня недели, а пол первого и второго ребенка независимы. Для каждой из ситуаций найдите условную вероятность того, что у тёти Маши есть дети обоих полов.
\begin{enumerate}
\item Известно, что старший ребенок — мальчик.
\item Тетя Маша наугад выбирает одного своего
ребенка и посылает к тете Оле, вернуть метлу. Это оказывается мальчик.
\item На вопрос: «А правда ли тётя Маша, что у Вас есть хотя бы один сын?» тётя Маша ответила: «Да».
\item На вопрос: «А правда ли тётя Маша, что у Вас есть хотя бы один сын, родившийся в пятницу?» тётя Маша ответила: «Да».
\end{enumerate}

\item Ты смертельно болен. Спасти тебя может только один вид  целебной лягушки. Целебны у этого вида только самцы. Самцы и самки встречаются равновероятно. Ты на дороге и предельно ослаб. Слева в 100 метрах от тебя одна лягушка целебного вида, но не ясно, самец или самка. Справа в 100 метров аж две лягушки целебного вида, но тоже издалека неясно кто. От двух лягушек в твою сторону дует ветер, поэтому ты можешь их слышать.

В какую сторону стоит ползти из последних сил в каждой из  ситуаций?
\begin{enumerate}
  \item Cамцы и самки квакают одинаково, со стороны правых двух лягушек ты слышишь кваканье.
  \item Самки квакают, самцы — нет, со стороны правых двух лягушек ты слышишь кваканье, но не разобрать, одной лягушки или двух.
  \item Самцы и самки квакают по разному, но одинаково часто. Ты слышишь отдельный квак одной из двух лягушек справа и это квак самки.
\end{enumerate}

\item Monty-Hall

Есть три закрытых двери. За двумя из них — по козе, за третьей автомобиль. Ты выбираешь одну из дверей. Допустим, ты выбрал дверь А. Ведущий шоу открывает дверь B и за ней нет автомобиля.
В этот момент ведущий предлагает тебе изменить выбор двери.

Имеет ли смысл изменить выбор в каждой из трёх ситуаций?
\begin{enumerate}
  \item Ведущий выбирал одну из трёх дверей равновероятно.
  \item Ведущий выбирал одну из двух дверей не выбранных тобой равновероятно.
  \item Ведущий выбирал дверь без машины и не совпадающую с твоей.
\end{enumerate}

\newpage
\section{Встреча 3}

Состав: 13 школьников и 1 вожатая

Настя с моей поддержкой разобрала 2г. Решили 3а, 3б, 3в.


\end{enumerate}

\end{document}
